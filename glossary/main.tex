%\makeglossaries
\makenoidxglossaries

% Danh mục thuật ngữ
\newglossaryentry{ml}
{
	name={machine learning},
	description={học máy hay máy học}
}

\newglossaryentry{traffic}{
	name = {traffic}, 
	description = {lưu lượng truy cập mạng, đề cập đến lượng dữ liệu di chuyển trên mạng tại một thời điểm nhất định}
}

\newglossaryentry{ai-gls}{
	name = {artificial intelligence},
	description = {trí tuệ nhân tạo hay trí thông minh nhân tạo}
}

\newglossaryentry{entropy}{
	name = {entropy},
	description = {độ hỗn loạn thông tin}
}

\newglossaryentry{information-gain}{
	name = {information gain},
	description = {độ lợi thông tin, lượng thông tin có lợi cho quá trình xây dựng cây quyết định}
}

\newglossaryentry{html}{
	name = {HMTL},
	description = {(viết tắt của Hyper Text Markup Language) là ngôn ngữ đánh dấu siêu văn bản}
}

\newglossaryentry{supervised-learning}{
	name = {supervised learning},
	description = {học có giám sát}
}

\newglossaryentry{unsupervised-learning}{
	name = {unsupervised learning},
	description = {học không giám sát}
}

\newglossaryentry{regression}
{
	name=regression,
	description={hồi quy}
}

\newglossaryentry{decisiontree}{
	name = {decision tree},
	description = {cây quyết định}
}

\newglossaryentry{bdecisiontree}{
	name = {binary decision tree},
	description = {cây quyết định nhị phân, tại mỗi nút có tối đa hai nút con}
}

\newglossaryentry{non-leafnode}{
	name = {non-leaf node},
	description = {nút trong, nút có con}
}

\newglossaryentry{leafnode}{
	name = {leaf node},
	description = {nút lá, nút thể hiện đầu ra, nút không có con}
}

\newglossaryentry{childnode}{
	name = {child node},
	description = {nút con}
}

\newglossaryentry{rootnode}{
	name = {root node},
	description = {nút gốc}
}

\newglossaryentry{sibling node}
{
	name={sibling node},
	description={các nút có cùng nút cha}
}

\newglossaryentry{classification}
{
	name=classification,
	description={phân loại}
}

\newglossaryentry{overfitting}{
	name = {overfitting},
	description = {quá khớp, khi mô hình quá khớp với dữ liệu dùng để xây dựng cây thì sẽ gặp sai số lớn với dữ liệu mới}
}

\newglossaryentry{underfitting}{
	name = {underfitting},
	description = {không khớp, khi mô hình xây dựng được khác xa so với thực tế}
}

\newglossaryentry{accuracy}{
	name = {accuracy},
	description = {độ chính xác, dự đoán độ chính xác của mô hình phân loại}
}

\newglossaryentry{label}{
	name = {label},
	description = {nhãn, giá trị chỉ ra loại mà điểm dữ liệu thuộc về}
}

\newglossaryentry{normal}{
	name = {normal},
	description = {bình thường}
}

\newglossaryentry{anormalous}{
	name = {anormalous},
	description = {bất thường}
}

\newglossaryentry{attribute}{
	name = {attribute},
	description = {thuộc tính}
}

\newglossaryentry{gini}{
	name = {gini index},
	description = {chỉ số đo lường độ không sạch, hỗn loạn của dữ liệu}
}

\newglossaryentry{request-line}{
	name = {request line},
	description = {chỉ dòng đầu tiên trong gói tin HTTP request, bao gồm phương thức, URI, phiên bản giao thức và kết thúc bằng CRLF}
}

\newglossaryentry{request-body}{
	name = {request body},
	description = {phần nội dung trong gói tin HTTP request}
}

\newglossaryentry{discrete}{
	name = {discrete},
	description = {rời rạc}
}

\newglossaryentry{continuous}{
	name = {continuous},
	description = {liên tục}
}

\newglossaryentry{random-variable}{
	name = {random variable},
	description = {biến ngẫu nhiên}
}

\newglossaryentry{indicatorvariable}{
	name = {indicator variable},
	description = {hay còn biết đến với tên gọi là biến nhị phân}
}

\newglossaryentry{joint-probobability}{
	name = {joint probobability},
	description = {xác xuất hợp, xác suất có hai biến cố cùng xảy ra}
}

\newglossaryentry{conditional-probability}{
	name = {conditional probability},
	description = {xác suất có điều kiện}
}

\newglossaryentry{training-data}{
	name = {training data},
	description = {dữ liệu huấn luyện}
}

\newglossaryentry{test-data}{
	name = {test data},
	description = {dữ liệu kiểm tra hay dữ liệu kiểm thử}
}

\newglossaryentry{outcome}{
	name = {outcome},
	description = {đầu ra của dữ liệu}
}

\newglossaryentry{leave-one-out}{
	name = {leave-one-out},
	description = {còn lại một, dùng để chỉ tập dữ liệu chỉ còn một phần tử}
}

\newglossaryentry{validation}{
	name = {validation},
	description = {một kĩ thuật để đánh giá độ chính xác của mô hình}
}

\newglossaryentry{cross-validation}{
	name = {cross validation},
	description = {một kĩ thuật để đánh giá độ chính xác của mô hình, tiên tiến hơn validation}
}

\newglossaryentry{train-error}{
	name = {train error},
	description = {mất mát trên dữ liệu huấn luyện}
}

\newglossaryentry{test-error}{
	name = {test error},
	description = {mất mát trên dữ liệu kiểm tra}
}

\newglossaryentry{validation-error}{
	name = {validation error},
	description = {mất mát trên tập validation}
}

\newglossaryentry{validation-set}{
	name = {validation set},
	description = {tập dữ liệu được trích ra từ tập huấn huyện dùng cho mục đích kiểm tra độ chính xác của mô hình}
}

\newglossaryentry{training-score}{
	name = {training score},
	description = {độ chính xác khi huấn luyện, xây dựng cây}
}

\newglossaryentry{crossvalidation-score}{
	name = {cross-validation score},
	description = {độ chính xác khi dự đoán trên tập cross-validation}
}


% Danh mục từ viết tắt
% Ví dụ
\newacronym{http}{HTTP}{HyperText Transfer Protocol}
\newacronym{ai}{AI}{Artificial Intelligence}
\newacronym{cart}{CART}{Classification and Regression Trees}
\newacronym{csdl}{CSDL}{Cơ Sở Dữ Liệu}
\newacronym{xss}{XSS}{Cross-Site Scripting}
\newacronym{id3}{ID3}{Iterative Dichotomiser 3}
\newacronym{csic}{CSIC}{Consejo Superior de Investigaciones Científicas}